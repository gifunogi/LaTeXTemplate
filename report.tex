\documentclass{jsarticle}
\usepackage{listings}			% 行番号等のリスト作成をする
\usepackage{jlisting}			% リストを日本語対応させる
\usepackage{multicol}			% 部分的に2段組にする
\usepackage[dvipdfmx]{graphicx}	% 図の挿入の補助
\usepackage{here}				% ここに表示するの!
\usepackage{amsmath,amssymb}    % 数学記号の補助
\usepackage{ascmac}             % 枠付き文章の補助
\usepackage{url}				% url出力の補助

% \usepackage{fancyhdr}         % ページ番号を右上に表示するとき
% \pagestyle{fancy}
% \lhead{}
% \rhead{}
% \rhead{\thepage{}}
% \cfoot{}
% \renewcommand{\headrulewidth}{0pt}

\makeatletter
\renewcommand{\theequation}{	    % 式番号の付け方
\thesection.\arabic{equation}}
\@addtoreset{equation}{section}
\renewcommand{\thefigure}{	        % 図番号の付け方
\thesection.\arabic{figure}}
\@addtoreset{figure}{section}
\renewcommand{\thetable}{		    % 表番号の付け方
\thesection.\arabic{table}}
\@addtoreset{table}{section}
\renewcommand{\refname}{}
\AtBeginDocument{			    	% リスト番号の付け方
\renewcommand*{\thelstlisting}{\arabic{section}.\arabic{lstlisting}}%
\@addtoreset{lstlisting}{section}}
\lstset{                            % listings の表示設定
    breaklines = true,              % 自動で折り返す。
    tabsize = 4,                    % インデントをスペースいくつで表すか
    frame = single,                 % 枠を上下左右に表示する
    basicstyle = \ttfamily\footnotesize,
    numbers = left,	                % 行番号を左に
    language = C                    % 言語設定
}

\makeatother	
\begin{document}

\title{タイトル}
\author{著者名}
\date{平成YY年MM月DD日}
\maketitle
% \begin{multicols}{2}				% 2段組にするときはじめ
% \end{multicols}{2}                % 2段組おわり
\section{セクション}
\subsection{サブセクション}
\subsubsection{サブサブセクション}
\section{参考文献}
\begin{thebibliography}{99}
\bibitem{quadratic}資料名\url{***}
\end{thebibliography}
\section{付録}
% ソースコード添付の例
\lstinputlisting[caption=フィボナッチ数出力プログラム, label=sec:fib]{./input/fib.c}
% 画像添付するとき
% \begin{figure}[H]
% \begin{center}
% \includegraphics[width=120mm]{./img/[ファイル名]}
% \caption{画像キャプション}
% \end{center}
% \end{figure}

\end{document}